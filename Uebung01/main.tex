\documentclass[DIN, pagenumber=false, fontsize=11pt, parskip=half]{scrartcl}

\usepackage{amsmath}
\usepackage{amsfonts}
\usepackage{amssymb}
\usepackage{enumitem}
\usepackage[utf8]{inputenc} % this is needed for umlauts
\usepackage[T1]{fontenc} 
\usepackage[ngerman]{babel}
\usepackage{commath}
\usepackage{xcolor}
\usepackage{booktabs}
\usepackage{float}
\usepackage{tikz-timing}
\usepackage{tikz}
\usepackage{multirow}
\usepackage{colortbl}
\usepackage{xstring}
\usepackage{circuitikz}
\usepackage{listings} % needed for the inclusion of source code
\usepackage[final]{pdfpages}
\usepackage{subcaption}
\usepackage{import}
\usepackage{cleveref}
\usepackage{bm}
\usepackage{url}

\usetikzlibrary{calc,shapes.multipart,chains,arrows}

\newcommand{\Prb}[1]{P(\text{#1})}
\newcommand{\CPr}[2]{P(\text{#1}|\text{#2})}
\DeclareMathOperator*{\argmax}{arg\,max}
\DeclareMathOperator*{\argmin}{arg\,min}
\DeclareMathOperator{\rank}{rank}
\newcommand{\R}[0]{\mathbb{R}}

%Inkscape fuckery
\newcommand{\incfig}[2][\columnwidth]{%
    \def\svgwidth{#1}
    \import{./}{#2.eps_tex}
}

\title{Entwurfsmethodik Eingebetteter Echtzeitsysteme}
\author{Paul Nykiel}

\begin{document}
    \maketitle
    \section{Raumflug}
    \begin{enumerate}[label=\alph*)]
        \item Bei der Auswahl einer geeigneten Trajektorie zu einem anderen Himmelskörper
            muss ein Kompromiss aus den Größen Flugzeit und Treibstoffverbrauch gefunden
            werden. Der direkte Weg (\textit{direct ascent}) ist die schnellste Variante
            braucht jedoch auch eine große Menge Treibstoff, folglich wird eine große
            Trägerrakete benötigt oder die Traglast ist drastisch reduziert. Bei einer
            Trajektorie welche einen (Teil-) Orbit um Erde und Mond beeinhaltet wird der
            Treibstoffverbrauch erheblich reduziert. Dadurch ist es möglich kleinere
            Trägerraketen zu nutzen und mehr Nutzlast mit sich zu führen (im Falle der
            Apollo Missionen die Mondlandefähre und der Mondorbiter).

            Da bereits die Saturn-V Rakete für damalige (und auch heutige) Verhältnisse
            sehr groß war, entschieden sich die Ingenieure wohl für eine 
            Hohmann-Trajektorie, in frühen Phasen der Entwicklung wurde jedoch auch ein
            Direkaufstieg evaluiert. \footnote{\url{https://en.wikipedia.org/wiki/Direct_ascent}}

            Ein weiter Vorteil der verwendeten Trajektorie ist es vor dem 
            \textit{trans lunar injection burn} die Funktionalität aller Teilsysteme
            im Erdorbit zu verifizieren. Ein \textit{direct ascent} macht einen 
            Missionsabruch erheblich komplizierter.
        \item Die Fluchtgeschwindigkeit der Erde beträgt $11186\frac{m}{s}$.
            Ab dieser Geschwindigkeit steigt die Distanz eines Körpers zur Ende
            kontinuierlich an: der Körper kann den Einflussbereich der Erdgravitation
            (asymptotisch) verlassen. Dies lässt sich durch Betrachtung der Energien
            erklären: ab der Fluchtgeschwindigkeit hat ein Körper mehr kinetische Energie
            als er braucht um auf \glqq{}unendliche\grqq{} Höhe angehoben zu werden.
            \footnote{\url{https://en.wikipedia.org/wiki/Escape_velocity}}
        \item Das Startgewicht beträgt $2900000$ kg. Der Schub der ersten Stufe beträgt
            33MN auf Meereshöhe und 40MN bei geringerem Luftdruck.
            \footnote{\url{https://en.wikipedia.org/wiki/Saturn_V}}
    
            
        \item Selbst wenn ein Körper ausreichend Energie hat um den Einflussbereich
            der Erde zu verlassen so ist er immer noch im Einflussbereich der Sonne.
            Die Sonne hat, aufgrund ihrer im Vergleich zur Erde deutlich größeren Masse,
            eine Fluchtgeschwindigkeit von $617500\frac{m}{s}$. Solange diese 
            Geschwindigkeit nicht erreicht ist bleibt der Körper in einem Sonnenorbit
            und verlässt folglich das Sonnensystem nicht dauerhaft.
        \item Der Satellit beginnt auf einem circulären Orbit um den Körper. Durch
            einen ersten Impulstoß wird dieser Orbit stark elliptisch, der Satellit 
            entfernt sich zuerst von dem Körper. Am Apogäum wird durch einen zweiten 
            Impulsstoß der Orbit wieder auf eine Kreisförmigebahn verformt. Der Satellit
            befindet sich nun wieder in einem circulären Orbit, jedoch auf einer höheren
            Umlaufbahn.
            \footnote{\url{https://en.wikipedia.org/wiki/Hohmann_transfer_orbit}}
    \end{enumerate}

    \section{Echtzeitanalyse}
    \begin{enumerate}[label=\alph*)]
        \item Der Artikel hat den Titel \glqq{}Scheduling Algorithms for Multiprogramming 
            in a Hard-Real-Time Environment\grqq{} und ist z.B. auf der Website der
            Universität von Radboud zu finden (\url{https://www.cs.ru.nl/~hooman/DES/liu-layland.pdf}).
        \item Der Artikel begint damit fünf zentrale Annahmen über das untersuchte,
            präemptive Multithreadingsystem zu machen:
            \begin{enumerate}
                \item Alle Tasks haben periodische Deadlines mit fixer Periode
                \item Deadlines beziehen sich nur darauf, dass ein Task bis zu diesem
                    Zeitpunkt beendet sein muss
                \item Die Tasks sind unabhängig
                \item Die Tasks haben jeweils konstante Laufzeit
                \item Nicht periodische Tasks kommen im Regelbetrieb nicht vor und
                    haben keine Echtzeitanforderungen
            \end{enumerate}
            Die Tasks werden mit $\tau_i$ bezeichnet, einzelnen Perioden mit 
            $T_i$ sowie die Laufzeiten mit $C_i$ (jeweils $i \in \{1, \ldots, m\}$).

            Unter diesen Annahmen werden dann drei Schedulingverfahren untersucht: ein
            statisches Schedulingverfahren, ein dynamisches Schedulingverfahren sowie
            ein hybrides Verfahren.

            Für das statische Verfahren wird festgestellt, dass alle Schedulingprobleme
            die statisch lösbar sind durch eine priorisierung anhand der Periodendauer
            gelöst werden. Folglich kann diese priorisierung genutzt werden um zu testen
            ob ein statisches Schedulingverfahren existiert, daher nennen die Autoren
            dieses Verfahren optimal. Abschließend bestimmen die Autoren die maximale
            Auslastung des Prozessors zu:
            \begin{equation}
                U = m \cdot \left(2^{\frac{1}{m}} -1 \right)
            \end{equation}

            Um diese Auslastung zu verbessern schlagen die Autoren ein dynamisches
            Schedulingverfahren vor. Hierbei wird die Priorität auf Basis der nächsten
            Deadline gewählt. Dieses Verfahren löst des Schedulingverfahren genau dann
            wenn
            \begin{equation}
                \sum_i \frac{C_i}{T_i} \leq 1
            \end{equation} 
            Hierbei erfüllt der Algorithmus das gleiche Optimalitätskriterium wie das
            statische Verfahren. Dieses Verfahren kann jedoch höhere Auslastungen
            erreichen.

            Als letztes Verfahren schlagen die Autoren ein hybrides Verfahren vor.
            Dies begründen sie dadurch, dass Prozessoren zur damaligen Zeit nicht
            fähig waren die priorisierung von Unterbrechungen dynamisch zu verändern.
            Folglich können diese Prozessoren auf Hardware-Ebene nur statische
            Schedulingverfahren verwenden. Dies resultiert im Ansatz die schnellsten
            $k$ Tasks (geordnet nach Periode $T_i$) statisch zu verwalten sowie
            die langsameren $m-k$ Tasks dynamisch zu verwalten, hierbei werden jeweils
            die oben vorgestellten Verfahren genutzt. Auch durch dieses Verfahren kann
            eine Lösung gefunden werden, die Auslastung ist jedoch geringer als bei der
            rein dynamischen Verwaltung.
    \end{enumerate}

    \section{Begriffe}
    \begin{enumerate}[label=\alph*)]
        \item Ein eingebettes System ist ein Computersystem eingebettet in einen 
            technischen Kontext
        \item Eingebettet Systeme haben besondere Anforderung, z.B. an die 
            Echtzeitfähigkeit aber auch an die Implementierung in Hardware,
            z.B. in Form von Energie-, Kosten- und Größenlimitierungen.
        \item Bei harten Echtzeitsystemen muss die Deadline zwingend immer eingehalten
            werden. Weiche Echtzeitsysteme erfordern nur, dass das Einhalten der
            Deadline einer bestimmten statistischen Verteilung folgt, z.B. dürfen
            nicht mehr als 10\% aller Deadlines verpasst werden.
        \item Ein \glqq{}Intellectual Property Core\grqq{} (IP-Core). Ist eine Komponente,
            diese kann sowohl in Hardware wie auch in Software implementiert sein,
            die eine bestimmte Funktion zur Verfügung stellt ohne dass der Anwender
            (z.B. der Entwickler des Gesamtsystems) wissen über die innere Funktionsweise
            der Komponente erlangen muss.
        \item Ein System-On-a-Chip (SoC) ist ein hochintegriertes eingebettetes System,
            dieses besteht nicht nur aus einem Prozessor sondern auch Peripherie wie
            Speicher (sowohl RAM als auch persistenter Speicher), Sensoren (oftmals in 
            MEMS-Bauform) und Kommunikationsschnittstellen (z.B. WLAN, Mobilfunk).
    \end{enumerate}

    \section{Zeit}
    \begin{enumerate}[label=\alph*)]
        \item Ein Sonnentag ist die Dauer zwischen zwei Meridiandurchgänge der Sonne
            \footnote{\url{https://de.wikipedia.org/wiki/Sonnentag}}. Da dieser Aufgrund
            des elliptischen Orbits der Erde um die Sonne leicht über den Verlauf eines
            Jahres vaariert gibt es einen mittleren Sonnetag.
        \item Es wird angenommen, dass es genau 12:00 wenn es in London bereits
            16:00 ist. Folglich beträgt die Zeitverschiebung 4 Stunden, da jede
            Stunde zu $15^\circ$ korrespondiert (da 
            $\frac{360^\circ}{24h} = 15\frac{\circ}{\text{h}}$)
            also befindet sich das Schiff auf $4 \cdot 15^\circ = 60^\circ$ westlicher
            Breite.

            In der Nähe befinden sich die Inseln Dominica und Martinique.
        \item Neben der oben vorgestellten Methode zur Bestimmung von Zeit auf Basis
            eines Sonnentages können auch andere periodische Prozesse genutzt werden.
            Angefangen von mechanischen, schwingungsfähigen System wie Pendel über
            elektronische Schwingkreise und Quarzoszillatoren.

            Inzwischen werden Atomuhren als Zeitreferenz genutzt, hierbei wird die
            Ressonanzfrequenz bestimmter Atome ausgenutzt und darüber eine Uhr
            synchronisiert \footnote{\url{https://de.wikipedia.org/wiki/Atomuhr}}.
            
            Eine technische Norm muss hierbei die Vorgehensweise und Parameter
            spezifizieren, so dass die Messung theoretisch von jeder Person reproduziert
            werden kann. Im Falle der Atomuhr sind das Parameter wie das verwendete
            Element, die Anzahl an Schwingungen die einer Zeiteinheit (z.B. Sekunde)
            entsprechen, sowie die Eigenschaften des Regelkreises der die Ressonanzfrequenz
            einregelt.

            Die Bürgerliche Zeit wird in Deutschland durch das Einheiten- und
            Zeitgesetz spezifiziert 
            \footnote{\url{https://de.wikipedia.org/wiki/Gesetzliche_Zeit}},
            dieses definiert die Zeit auf Basis der koordinierten Weltzeit (UTC),
            welche durch das Internationale Büro für Maß und Gewicht bestimmt wird
            \footnote{\url{https://de.wikipedia.org/wiki/Koordinierte_Weltzeit}}.
            Für die Bestimmung der UTC wird internationale Atomzeit (TAI) genutzt
            welche aus 600 Atomuhren als gewichtetes Mittel gebildet wird
            \footnote{\url{https://de.wikipedia.org/wiki/Internationale_Atomzeit}}. 
    \end{enumerate}
    
\end{document}

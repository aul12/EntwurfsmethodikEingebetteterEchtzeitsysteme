\documentclass[DIN, pagenumber=false, fontsize=11pt, parskip=half]{scrartcl}

\usepackage{amsmath}
\usepackage{amsfonts}
\usepackage{amssymb}
\usepackage{enumitem}
\usepackage[utf8]{inputenc} % this is needed for umlauts
\usepackage[T1]{fontenc} 
\usepackage[ngerman]{babel}
\usepackage{commath}
\usepackage{xcolor}
\usepackage{booktabs}
\usepackage{float}
\usepackage{tikz-timing}
\usepackage{tikz}
\usepackage{multirow}
\usepackage{colortbl}
\usepackage{xstring}
\usepackage{circuitikz}
\usepackage{listings} % needed for the inclusion of source code
\usepackage[final]{pdfpages}
\usepackage{subcaption}
\usepackage{import}
\usepackage{cleveref}
\usepackage{bm}
\usepackage{url}
\usepackage{pgfplots}
\usepackage{mathtools}


\usetikzlibrary{calc,shapes.multipart,chains,arrows}

\newcommand{\Prb}[1]{P(\text{#1})}
\newcommand{\CPr}[2]{P(\text{#1}|\text{#2})}
\DeclareMathOperator*{\argmax}{arg\,max}
\DeclareMathOperator*{\argmin}{arg\,min}
\DeclareMathOperator{\rank}{rank}
\newcommand{\R}[0]{\mathbb{R}}

%Inkscape fuckery
\newcommand{\incfig}[2][\columnwidth]{%
    \def\svgwidth{#1}
    \import{./}{#2.eps_tex}
}

\title{Entwurfsmethodik Eingebetteter Echtzeitsysteme}
\author{Paul Nykiel}

\begin{document}
    \maketitle
    \section{Begriffe}
    \begin{enumerate}[label=\alph*)]
        \item Die Antwortzeit beschreibt die Zeit zwischen einem Ereignis und
            der Reaktion des (Eingebetteten-) Systems auf dieses Ereignis. Die
            Ausführungszeit ist in diesem Kontext ein Teil der Antwortzeit, und zwar
            jene Zeit in der das eingebettete System das Ereignis verarbeitet und eine
            Antwort bestimmt (also die Laufzeit des Algorithmus). Weitere Komponenten
            der Antwortzeit ist unter anderem die Zeit die das eigebettete System
            braucht bis mit der Ausführung des Algorithmus beginnt, diese wird
            vor allem durch den Scheduler beeinflusst. 
        \item Ein Kontrollflussgraph ist ein Graph, welcher sich auf den Programablauf
            fokusiert. Dabei symbolisieren Knoten Anweisungen oder Kontrollstrukturen
            und Kanten zeigen den Ablauf des Program. Ein Datenflussgraph ist die hierzu
            duale Darstellung: hierbei wird sich auf die Verarbeitung von Daten 
            konzentriert. Die Knoten des Graphens sind Operationen und Funktionen,
            die Kanten des Graphens symbolisieren jeweils ein Datum.

            Ein Grundblock ist ein Element eines Programs bei dem der Programfluss
            nicht verzweigen kann. Es kann sich also um Operationen, Funktionen
            oder auch Unterprogramme handeln, jedoch nicht um bedingte Sprünge.
    \end{enumerate}

    \section{Uhr} 
    \begin{enumerate}[label=\alph*)]
        \item
            Eine Allgemeine Uhr wird beschrieben durch:
            \begin{equation}
                c(t) = \left\lfloor \frac{(1+g) t}{T} \right\rfloor G \mod M
            \end{equation}
            mit
            \begin{itemize}
                \item $g$: Gangfehler (Abweichung der Taktfrequenz von der korrekten
                    Frequenz)
                \item $T$: Auflösung (Zeit zwischen Uhrschritten)
                \item $G$: Schrittweite (Inkrement pro Uhrschritt)
                \item $M$: Periode (Zeit bis zur Wiederholung)
            \end{itemize}
        \item 
            Siehe Abbildung \ref{fig:kennlinien}.
            \begin{figure}[h]
                \centering
                \begin{tikzpicture}
                    \begin{axis}[
                                xlabel=$t$,
                                ylabel=$c(t)$,
                                domain=0:10
                            ]
                        \addplot[color=red]{0.9*x};
                        \addplot[color=green]{x};
                        \addplot[color=blue]{1.1*x};
                    \end{axis}
                \end{tikzpicture}
                \caption{Kennlinien $c(t)$ dreier verschiedener Uhren:
                    ideal (\textcolor{green}{grün}), zu schnell (\textcolor{blue}{blau}) 
                    und zu langsam (\textcolor{red}{rot}).
                    Es wird angenommen, dass die Uhren bei $t=0$ synchronisiert werden.}
                \label{fig:kennlinien}
            \end{figure}
        \item Minimale Frequenz bei $f=1$kHz und einer Frequenzabweichung von $30$ppm:
            \begin{equation}
                f_\text{min} = f \cdot \left(1 - \frac{30}{{10}^6}\right) = 999.97\text{Hz}
            \end{equation}
            Maximale Frequenz:
            \begin{equation}
                f_\text{max} = f \cdot \left(1 + \frac{30}{{10}^6}\right) = 1000.03\text{Hz}
            \end{equation}
            Differenzfrequenz:
            \begin{equation}
                \Delta f = f_\text{max} - f_\text{min} = 0.06 \text{Hz}
            \end{equation}

            Anzahl an Takten:
            \begin{eqnarray}
                N_{1000\text{s}} &=& \Delta f \cdot 1000 \text{s} = 60 \\
                N_{1 \text{Tag} } &=& \Delta f \cdot 86400 \text{s} = 5184 
            \end{eqnarray}
        \item 
            Bei einem Frequenzfehler von $\pm30$ppm beträgt der Zeitfehler
            $\pm \frac{30 \text{ms}}{\text{s}}$ relativ zur korrekten Zeit,
            bzw. $\pm \frac{60 \text{ms}}{\text{s}}$ zwischen den zwei Systemen.
            Dieses Verhalten ist in Abbildung \ref{fig:skew} visualisiert.
            
            \begin{figure}[h]
                \centering
                \begin{tikzpicture}
                    \begin{axis}[
                                xlabel=$t$,
                                ylabel=$\abs{\Delta t}$,
                                domain=0:1000
                            ]
                        \addplot[color=black]{0.03*x};
                    \end{axis}
                \end{tikzpicture}
                \caption{Maximale betragsmäßige Zeitabweichung, es wird eine
                    Synchronisierung bei $t=0$ angenommen.}
                \label{fig:skew}
            \end{figure}
        \item
            Wenn die beiden Prozessoren über eine Schnittstelle welche keine
            separate Taktsynchronisierung unterstützt kommunizieren so kann die
            Frequenzabweichung zu Fehlern bei der Kommunikation führen, da die
            Daten zu unterschiedlichen Zeitpunkten geschrieben wie gelesen werden.
    \end{enumerate}
\end{document}

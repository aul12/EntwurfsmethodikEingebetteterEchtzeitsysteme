\documentclass[DIN, pagenumber=false, fontsize=11pt, parskip=half]{scrartcl}

\usepackage{amsmath}
\usepackage{amsfonts}
\usepackage{amssymb}
\usepackage{enumitem}
\usepackage[utf8]{inputenc} % this is needed for umlauts
\usepackage[T1]{fontenc} 
\usepackage[ngerman]{babel}
\usepackage{commath}
\usepackage{xcolor}
\usepackage{booktabs}
\usepackage{float}
\usepackage{tikz-timing}
\usepackage{tikz}
\usepackage{multirow}
\usepackage{colortbl}
\usepackage{xstring}
\usepackage{circuitikz}
\usepackage{listings} % needed for the inclusion of source code
\usepackage[final]{pdfpages}
\usepackage{subcaption}
\usepackage{import}
\usepackage{cleveref}
\usepackage{bm}
\usepackage{url}
\usepackage{pgfplots}
\usepackage{mathtools}


\usetikzlibrary{calc,shapes.multipart,chains,arrows}

\newcommand{\Prb}[1]{P(\text{#1})}
\newcommand{\CPr}[2]{P(\text{#1}|\text{#2})}
\DeclareMathOperator*{\argmax}{arg\,max}
\DeclareMathOperator*{\argmin}{arg\,min}
\DeclareMathOperator{\rank}{rank}
\newcommand{\R}[0]{\mathbb{R}}

%Inkscape fuckery
\newcommand{\incfig}[2][\columnwidth]{%
    \def\svgwidth{#1}
    \import{./}{#2.eps_tex}
}

\title{Entwurfsmethodik Eingebetteter Echtzeitsysteme}
\author{Paul Nykiel}

\begin{document}
    \maketitle
    \section{Matlab}
    \begin{enumerate}[label=\alph*)]
        \item Upper Heaviside:
            \lstinputlisting[language=Matlab]{upperHeaviside.m}
        \item Lower Heaviside:
            \lstinputlisting[language=Matlab]{lowerHeaviside.m}
        \item Kronecker-Delta:
            \lstinputlisting[language=Matlab]{kronDelta.m}
    \end{enumerate}

    \section{Entwurfsablauf}
    \begin{enumerate}[label=\alph*)]
        \item Der Entwursablauf beginnt mit der Spezifikation, diese bestimmt die Aufgabe des ES, z.B.
            in Form eines Algorithmus. Mit Bauteilen (sowohl Hardware als auch Software) aus der
            Plattformbibliothek wird dann ein erstes Systemmodell erstellt. Aus diesem Modell werden
            Parameter wie Ausführungszeiten und Energieverbrauch mithilfe einer intrinsischen Analyse
            extrahiert. Die einzelnen Parameter werden dann in der extrinsischen Analyse um Parameter
            des Gesamtssystems, z.B. die Antwortzeit und den gesamten Energieverbrauch zu bestimmen.
            
            Die Parameter des Gesamtsystems werden dann als Basis für eine Optimierung genutzt: die Architektur
            und die Parameter der Systemkomponenten werden angepasst bis die optimale Systemarchitektur mit den
            optimalen Parametern gefunden wurde.

            Das optimale System (also das Ergebnis der Systemsynthese) wird dann implementiert, dabei können
            Teilsysteme in Hardware und andere Teilsysteme in Software implementiert werden, je nach Resultat
            der Optimierung.
        \item Die Plattformbibliothek stellt fertige, aber parametrierbare Komponenten zur Verfügung aus denen
            das Gesamtsystem aufgebaut werden kann. Diese Komponenten können in Hardware und Software implementiert
            sein.
        \item Die intrinsiche Analyse dient dem Extrahieren der Parameter aus den Teilsystemen. So kann beispielsweise
            die Ausführungszeit eines (Teil-)Algorithmus (z.B. FFT) in einem der Komponenten (z.B. DSP), unter 
            Berücksichtigung der gegebenen Parameter (z.B. Signallänge) bestimmt werden.
        \item Bei der extrinsichen Analyse wird das Gesamtsystem \glqq{}von außen\grqq{} betrachtet. Die intrinsiche
            Analyse betrachtet das Verhalten der einzelnen Teilkomponenten. 
        \item Bei der Optimierung wird eine Gesamtsystemarchitektur als Kombination der Einzelsysteme bestimmt,
            so dass das Gesamtsystem nach einem gegebenen Gütemaß optimal ist. Dabei werden verschiedene Ressourcen
            (Allokation); verschiedene Aufteilung von Aufgaben an die Ressourcen (Bindung); und verschiedene
            Schemata der Aufgabenbearbeitung (Ablaufplanung) getestet.
    \end{enumerate}
\end{document}

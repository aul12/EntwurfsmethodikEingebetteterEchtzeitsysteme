\documentclass[DIN, pagenumber=false, fontsize=11pt, parskip=half]{scrartcl}

\usepackage{amsmath}
\usepackage{amsfonts}
\usepackage{amssymb}
\usepackage{enumitem}
\usepackage[utf8]{inputenc} % this is needed for umlauts
\usepackage[T1]{fontenc} 
\usepackage[ngerman]{babel}
\usepackage{commath}
\usepackage{xcolor}
\usepackage{booktabs}
\usepackage{float}
\usepackage{tikz-timing}
\usepackage{tikz}
\usepackage{multirow}
\usepackage{colortbl}
\usepackage{xstring}
\usepackage{circuitikz}
\usepackage{listings} % needed for the inclusion of source code
\usepackage[final]{pdfpages}
\usepackage{subcaption}
\usepackage{import}
\usepackage{cleveref}
\usepackage{bm}
\usepackage{url}
\usepackage{pgfplots}
\usepackage{mathtools}

\DeclarePairedDelimiter\ceil{\lceil}{\rceil}
\DeclarePairedDelimiter\floor{\lfloor}{\rfloor}


\usetikzlibrary{calc,shapes.multipart,chains,arrows}

\newcommand{\Prb}[1]{P(\text{#1})}
\newcommand{\CPr}[2]{P(\text{#1}|\text{#2})}
\DeclareMathOperator*{\argmax}{arg\,max}
\DeclareMathOperator*{\argmin}{arg\,min}
\DeclareMathOperator{\rank}{rank}
\newcommand{\R}[0]{\mathbb{R}}

%Inkscape fuckery
\newcommand{\incfig}[2][\columnwidth]{%
    \def\svgwidth{#1}
    \import{./}{#2.eps_tex}
}

\title{Entwurfsmethodik Eingebetteter Echtzeitsysteme}
\author{Paul Nykiel}

\begin{document}
    \maketitle
    \section{Kontroll und Datenflussgraphen}
    Siehe Abbildung \ref{fig:control} und \ref{fig:data}.
    \begin{figure}[h]
        \centering
        \includegraphics[width=\textwidth]{control.pdf}
        \caption{Kontrollflussgraph}
        \label{fig:control}
    \end{figure}
    \begin{figure}[h]
        \centering
        \includegraphics[width=\textwidth]{data.pdf}
        \caption{Datenflussgraph}
        \label{fig:data}
    \end{figure}
        
    \section{Matlab}
    \begin{enumerate}[label=\alph*)]
        \item Aufgrund der unendlichen Summe kann der gegebene Ausdruck nicht direkt
            mit Matlab implementiert werden. Deshalb folgt eine kurze Analyse
            des Arguments der Summe:

            Der erste Heaviside Ausdruck ist genau dann $1$ falls
            \begin{equation}
                t - np \geq 0 \Leftrightarrow t \geq n p
            \end{equation}
            sonst ist der Ausdruck $0$. Analog ist der zweite Heaviside Ausdruck
            genau dann $1$ falls
            \begin{equation}
                np -t_a \geq 0 \Leftrightarrow np \geq t_a
            \end{equation}
            sonst ist der Ausdruck ebenfalls $0$. Für den dritten Heaviside Ausdruck
            gilt, er ist genau dann $1$ falls
            \begin{equation}
                t_b - n p > 0 \Leftrightarrow t_b > n p
            \end{equation}
            sonst ebenfalls $0$ (im Vergleich zu den anderen beiden Ausdrücken
            wird hier eine Lower Heaviside Funktion verwendet, daher das $>$ anstatt
            $\geq$).

            Damit das Argument der Summe nicht $0$ ist, müssen alle drei Heaviside
            Ausdrücke $\neq 0$, in diesem Fall also $=1$ sein. Das heißt es muss
            gelten:
            \begin{equation}
                t \geq n p \land np \geq t_a \land t_b > n p
            \end{equation}
            beziehungsweise
            \begin{equation}
                np \in [t_a, t_b) \land np \leq t
            \end{equation}
            Division durch $p$ ergibt:
            \begin{equation}
                n \in \left[ \frac{t_a}{p}, \frac{t_b}{p}\right) \land n \leq \frac{t}{p} 
            \end{equation}
            Somit ist die untere Grenze der Summe gegeben durch:
            \begin{equation}
                n^- = \ceil*{ \frac{t_a}{p} }
            \end{equation}
            die obere Grenze ist gegeben durch (die unterschiedlichen Ausdrücke
            kommen durch die Unterscheidung zwischen $<$ und $\leq$):
            \begin{equation}
                n^+ = \min \left(
                    \ceil*{\frac{t_b}{p}} - 1, 
                    \floor*{\frac{t}{p}}
                 \right)
            \end{equation}
            
            Da das Argument der Summe für diesen Fall immer $1$ ist lässt sich somit
            $\mathbb{E}$ schreiben als (vorrausgesetzt $n^+ \geq n^-$, sonst
            ist die Summe $0$):
            \begin{equation}
                \mathbb{E}(t, \Delta^a_b) = n^+ - n^- + 1
            \end{equation}

            Die Implementierung in Matlab besteht aus dieser geschlossenen Form:
            \lstinputlisting[language=Matlab,caption=\texttt{eventBound.m}]{eventBound.m}
        \item
            Es wurden verschiedene Werte mit folgendem Skript getestet:
            \lstinputlisting[language=Matlab, caption=\texttt{main.m}, label=lst:main]{main.m}

            mit denen im Skript angegebenen Werten wurde der Plot in Abbildung
            \ref{fig:eventBound} erzeugt.

            \begin{figure}[h]
                \centering
                \includegraphics[width=\textwidth]{eventBound.eps}
                \caption{Matlab Plot zu Listing \ref{lst:main}}
                \label{fig:eventBound}
            \end{figure}
    \end{enumerate}
\end{document}

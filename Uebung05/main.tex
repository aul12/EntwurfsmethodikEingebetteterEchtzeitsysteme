\documentclass[DIN, pagenumber=false, fontsize=11pt, parskip=half]{scrartcl}

\usepackage{amsmath}
\usepackage{amsfonts}
\usepackage{amssymb}
\usepackage{enumitem}
\usepackage[utf8]{inputenc} % this is needed for umlauts
\usepackage[T1]{fontenc} 
\usepackage[ngerman]{babel}
\usepackage{commath}
\usepackage{xcolor}
\usepackage{booktabs}
\usepackage{float}
\usepackage{tikz-timing}
\usepackage{tikz}
\usepackage{multirow}
\usepackage{colortbl}
\usepackage{xstring}
\usepackage{circuitikz}
\usepackage{listings} % needed for the inclusion of source code
\usepackage[final]{pdfpages}
\usepackage{subcaption}
\usepackage{import}
\usepackage{cleveref}
\usepackage{bm}
\usepackage{url}
\usepackage{pgfplots}
\usepackage{mathtools}

\DeclarePairedDelimiter\ceil{\lceil}{\rceil}
\DeclarePairedDelimiter\floor{\lfloor}{\rfloor}


\usetikzlibrary{calc,shapes.multipart,chains,arrows}

\newcommand{\Prb}[1]{P(\text{#1})}
\newcommand{\CPr}[2]{P(\text{#1}|\text{#2})}
\DeclareMathOperator*{\argmax}{arg\,max}
\DeclareMathOperator*{\argmin}{arg\,min}
\DeclareMathOperator{\rank}{rank}
\newcommand{\R}[0]{\mathbb{R}}

%Inkscape fuckery
\newcommand{\incfig}[2][\columnwidth]{%
    \def\svgwidth{#1}
    \import{./}{#2.eps_tex}
}

\title{Entwurfsmethodik Eingebetteter Echtzeitsysteme}
\author{Paul Nykiel}

\begin{document}
    \maketitle
    \section{Ereignissequenzen}
    Die gewählten Parameter werden, falls es nicht eindeutig ist, durch die entsprechenden
    Buchstaben mit Tilde bezeichnet. Die Parameter aus der Aufgabe haben nie Tilden.
    \begin{enumerate}[label=\roman*)]
        \item
            \begin{enumerate}[label=\alph*)]
                \item Periodische Ereignissequenz: $\tilde{p}=p$
                \item PeriodischeEreignissequenz mit Jitter: $\tilde{p}=p$, Jitter: $j=0$
                \item Sporadische, periodische Ereignissequenz: $p_i=0, p_0=p, n=1$
                \item Periodische Ereignissequenz mit initialem Schub: $\tilde{p}=p=\tilde{a}, j=0$
            \end{enumerate} 
        \item
            \begin{enumerate}[label=\alph*)]
                \item Periodische Ereignissequenz: $\tilde{p}=a$
                \item Periodische Ereignissequenz mit Jitter:
                    
                    Es wird ein Modell mit Jitter gewählt, hierfür wähle $\tilde{p}=a+\varepsilon$,
                    $\tilde{j}=3 \varepsilon$, diese Wahl der Parameter garantiert, dass
                    die Ereignisdichte immer eine obere Schranke ist.

                    Betrachte die Ereignisdichte als Funktion von $\varepsilon$:
                    \begin{equation}
                        \eta_{t,a}(\varepsilon) 
                            = \floor*{\frac{t+6 \varepsilon}{a+\varepsilon}}+1
                    \end{equation}
                    Die Funktion hat ihr minimum bei $\varepsilon=0$. Folglich
                    ist die kleinste obere Schranke für das Modell
                    Periodische Ereignissequenz mit Jitter für $\tilde{p}=a$ und $\tilde{j}=0$ gegeben.
                \item Sporadische, periodische Ereignissequenz: $p_i=a, p_0=p, n=5$
                \item Periodische Ereignissequenz mit initialem Schub: 
                    $\tilde{p}=p \tilde{a}=a, \tilde{j}=4\cdot p - t$
            \end{enumerate}
        \item
            \begin{enumerate}[label=\alph*)]
                \item Periodische Ereignissequenz: $\tilde{p}=a$
                \item PeriodischeEreignissequenz mit Jitter: 

                    Die beiden Randwerte des Optimierungsproblems sind bei
                    $\tilde{p}=a$ und $\tilde{p}=\frac{p}{3}$, $\tilde{j}$ ist gegeben
                    durch $\tilde{j} = 2 \tilde{p} - 2 a$. Eine Optimierung der
                    Ereignisdichte $\eta_{t,a}(p)$ führt zu einem Minimum bei
                    $\tilde{p}=a, \tilde{j}=0$.
                \item Sporadische, periodische Ereignissequenz: $p_i=a, p_0=p, n=3$
                \item Periodische Ereignissequenz mit initialem Schub: 

                    Da bis ins unendliche Ereignisse mit Abstand $a$ auftreten wähle
                    die Parameter zu: $\tilde{p}=a, \tilde{a}=a, \tilde{j}=0$. Dann ergibt
                    sich ein periodisches Modell mit Periode $a$.
            \end{enumerate} 
    \end{enumerate}
\end{document}

\documentclass[DIN, pagenumber=false, fontsize=11pt, parskip=half]{scrartcl}

\usepackage{amsmath}
\usepackage{amsfonts}
\usepackage{amssymb}
\usepackage{enumitem}
\usepackage[utf8]{inputenc} % this is needed for umlauts
\usepackage[T1]{fontenc} 
\usepackage[ngerman]{babel}
\usepackage{commath}
\usepackage{xcolor}
\usepackage{booktabs}
\usepackage{float}
\usepackage{tikz-timing}
\usepackage{tikz}
\usepackage{multirow}
\usepackage{colortbl}
\usepackage{xstring}
\usepackage{circuitikz}
\usepackage{listings} % needed for the inclusion of source code
\usepackage[final]{pdfpages}
\usepackage{subcaption}
\usepackage{import}
\usepackage{cleveref}
\usepackage{bm}
\usepackage{url}
\usepackage{pgfplots}
\usepackage{mathtools}

\DeclarePairedDelimiter\ceil{\lceil}{\rceil}
\DeclarePairedDelimiter\floor{\lfloor}{\rfloor}


\usetikzlibrary{calc,shapes.multipart,chains,arrows}

\newcommand{\Prb}[1]{P(\text{#1})}
\newcommand{\CPr}[2]{P(\text{#1}|\text{#2})}
\DeclareMathOperator*{\argmax}{arg\,max}
\DeclareMathOperator*{\argmin}{arg\,min}
\DeclareMathOperator{\rank}{rank}
\newcommand{\R}[0]{\mathbb{R}}

%Inkscape fuckery
\newcommand{\incfig}[2][\columnwidth]{%
    \def\svgwidth{#1}
    \import{./}{#2.eps_tex}
}

\title{Entwurfsmethodik Eingebetteter Echtzeitsysteme}
\author{Paul Nykiel}

\begin{document}
    \maketitle
    \section{Ereignissequenzen}
    Die gewählten Parameter werden, falls es nicht eindeutig ist, durch die entsprechenden
    Buchstaben mit Tilde bezeichnet. Die Parameter aus der Aufgabe haben nie Tilden.
    \begin{enumerate}[label=\roman*)]
        \item
            \begin{enumerate}[label=\alph*)]
                \item Periodische Ereignissequenz: $\tilde{p}=p$
                \item PeriodischeEreignissequenz mit Jitter: $\tilde{p}=p$, Jitter: $j=0$
                \item Sporadische, periodische Ereignissequenz: $p_i=0, p_0=p, n=1$
                \item Periodische Ereignissequenz mit initialem Schub: $\tilde{p}=p=\tilde{a}, j=0$
            \end{enumerate} 
        \item
            \begin{enumerate}[label=\alph*)]
                \item Periodische Ereignissequenz: $\tilde{p}=a$
                \item Periodische Ereignissequenz mit Jitter:
                    
                    Es wird ein Modell mit Jitter gewählt, hierfür wähle $\tilde{p}=a+\varepsilon$,
                    $\tilde{j}=3 \varepsilon$, diese Wahl der Parameter garantiert, dass
                    die Ereignisdichte immer eine obere Schranke ist.

                    Betrachte die Ereignisdichte als Funktion von $\varepsilon$:
                    \begin{equation}
                        \eta_{t,a}(\varepsilon) 
                            = \floor*{\frac{t+6 \varepsilon}{a+\varepsilon}}+1
                    \end{equation}
                    Die Funktion hat ihr minimum bei $\varepsilon=0$. Folglich
                    ist die kleinste obere Schranke für das Modell
                    Periodische Ereignissequenz mit Jitter für $\tilde{p}=a$ und $\tilde{j}=0$ gegeben.
                \item Sporadische, periodische Ereignissequenz: $p_i=a, p_0=p, n=5$
                \item Periodische Ereignissequenz mit initialem Schub: 
                    $\tilde{p}=p \tilde{a}=a, \tilde{j}=4\cdot p - t$
            \end{enumerate}
        \item
            \begin{enumerate}[label=\alph*)]
                \item Periodische Ereignissequenz: $\tilde{p}=a$
                \item PeriodischeEreignissequenz mit Jitter: 

                    Die beiden Randwerte des Optimierungsproblems sind bei
                    $\tilde{p}=a$ und $\tilde{p}=\frac{p}{3}$, $\tilde{j}$ ist gegeben
                    durch $\tilde{j} = 2 \tilde{p} - 2 a$. Eine Optimierung der
                    Ereignisdichte $\eta_{t,a}(p)$ führt zu einem Minimum bei
                    $\tilde{p}=a, \tilde{j}=0$.
                \item Sporadische, periodische Ereignissequenz: $p_i=a, p_0=p, n=3$
                \item Periodische Ereignissequenz mit initialem Schub: 

                    Da bis ins unendliche Ereignisse mit Abstand $a$ auftreten wähle
                    die Parameter zu: $\tilde{p}=a, \tilde{a}=a, \tilde{j}=0$. Dann ergibt
                    sich ein periodisches Modell mit Periode $a$.
            \end{enumerate} 
    \end{enumerate}

    \section{Intrinsische Analyse}
    \begin{figure}[h]
        \centering
        \includegraphics[width=\textwidth]{control.pdf}
        \caption{Kontrollflussgraph mit benannten Kanten und Basisblöcken
                aus dem vorherigen Übungsblatt}
    \end{figure}

    \begin{enumerate}[label=\alph*)]
        \item
            Für die Kanten gilt:
            \begin{eqnarray}
                d_1 + d_5 + d_6 &=& d_2 + d_7 \\
                d_2 &=& d_3 + d_4 \\
                d_3 &=& d_5 \\
                d_6 &=& d_4 \\
                d_7 &=& d_8 + d_9
            \end{eqnarray} 

            Für die Beziehung zwischen Knoten und Kanten gilt:
            \begin{eqnarray}
                x_1 &=& d_1 \\
                x_2 &=& d_1 + d_5 + d_6 \\
                x_3 &=& d_2 \\
                x_4 &=& d_3 \\
                x_4 &=& d_5 \\
                x_5 &=& d_4 \\
                x_5 &=& d_6 \\
                x_6 &=& d_7 \\
                x_7 &=& d_8 \\
                x_8 &=& d_9
            \end{eqnarray}

            Damit lassen sich die strukturellen Randbedingungen formulieren:
            \begin{eqnarray}
                x_1 &=& x_2 \\
                x_2 &=& x_3 + x_6 \\
                x_3 &=& x_4 + x_5 \\
                x_6 &=& x_2 \\
                x_6 &=& x_7 + x_8
            \end{eqnarray}
        \item
            Betrachtung der While-Schleife:
            
            Für den Fall $b=b_1=0$ wird die Schleifen gar nicht durchlaufen,
            folglich gilt $x_2 \geq 1$. Das Maximum wird sowohl für
            $a_1 = 65535, b_1 = 1$ sowie für $a_1=1, b_1=65535$ erreicht, beide
            Fälle führen den Schleifenkörper 65535 mal aus, somit wird $x_2$
            maximal 65536 mal geprüft. Also gilt:
            \begin{equation}
                1 \leq x_2 \leq 65536
            \end{equation}

            Außerdem lässt sich aus den oben genannten Fällen minima und maxima für
            $x_4$ und $x_5$ ableiten:
            \begin{eqnarray}
                0 \leq &x_4& \leq 65534 \\
                0 \leq &x_5& \leq 65535
            \end{eqnarray}

            Zudem gilt:
            \begin{eqnarray}
                x_1 &=& 1 \\
                x_6 &=& 1
            \end{eqnarray}
            und
            \begin{eqnarray}
                0 \leq &x_7& \leq 1  \\ 
                0 \leq &x_8& \leq 1 
            \end{eqnarray}
        \item
            Es gilt:
            \begin{eqnarray}
                c^- &=& c_1 \cdot 1 \\
                    &&+ c_2 \cdot 1 \\ 
                    &&+ c_3 \cdot 0 \\
                    &&+ c_4 \cdot 0 \\  
                    &&+ c_5 \cdot 0 \\
                    &&+ c_6 \cdot 1 \\
                    &&+ c_7 \cdot 0 \\
                    &&+ c_8 \cdot 1 \\
                    &=& c_1 + c_2 + c_6 + c_8
            \end{eqnarray}
            und
            \begin{eqnarray}
                c^+ &=& c_1 \cdot 1 \\
                    &&+ c_2 \cdot 65536 \\ 
                    &&+ c_3 \cdot 65535 \\
                    &&+ c_4 \cdot 0 \\  
                    &&+ c_5 \cdot 65535 \\
                    &&+ c_6 \cdot 1 \\
                    &&+ c_7 \cdot 1 \\
                    &&+ c_8 \cdot 0 \\
                    &=& c_1 + 65536 \cdot c_2 + 65535 \cdot c_3 + 65535 \cdot c_5 + c_6 + c_7
            \end{eqnarray}
        \item 
            Lauzeiten der Basisblöcke in Takten:
            \begin{eqnarray}
                c_1 &=& 1 + 1 = 2 \\
                c_2 &=& 1 \\
                c_3 &=& 1 \\
                c_4 &=& 1 \\
                c_5 &=& 1 \\
                c_6 &=& 1 \\
                c_7 &=& 6 + 10 = 16\\
                c_8 &=& 0
            \end{eqnarray}
            
            Damit ergeben sich die Ausführungszeiten zu:
            \begin{eqnarray}
                c^- &=& 2 + 1 + 1 + 0 = 4 \\
                c^+ &=& 2 + 65536 \cdot 1 + 65535 \cdot 1 + 65535 \cdot 1 + 1 + 16
                        = 196625
            \end{eqnarray}
    \end{enumerate}

    \section{Lineares Optimierungsproblem}
    \begin{enumerate}[label=\alph*)]
        \item
            Slack-Variablen Einfügen:
            \begin{eqnarray}
                2 x_1 + 3 x_2 + 2 x_3 + s_1 &=& 1000 \\
                x_1 + x_2 + 2 x_3 + s_2 &=& 800
            \end{eqnarray}
            mit $s_1, s_2 \geq 0$.

            Daraus folgt:
            \begin{eqnarray}
                A &=& \begin{pmatrix}
                    2 & 3 & 2 & 1 & 0 \\
                    1 & 1 & 2 & 0 & 1 \\
                \end{pmatrix} \\
                x &=& \begin{pmatrix}
                    x_1 \\ x_2 \\ x_3 \\ s_1 \\ s_2
                \end{pmatrix} \\
                b &=& \begin{pmatrix}
                    1000 \\ 800
                \end{pmatrix} \\
                c^\text{T} &=& \begin{pmatrix}
                    7 & 8 & 10 & 0 & 0
                \end{pmatrix}
            \end{eqnarray}

            Das Tableau ist gegeben durch
            \begin{equation}
                \begin{pmatrix}
                    2 & 3 & 2 & 1 & 0 & 1000 \\
                    1 & 1 & \mathbf{2} & 0 & 1 & 800  \\
                    7 & 8 & 10 & 0 & 0 & 0
                \end{pmatrix}
            \end{equation} 
            
            Das Maximum (4400) wird für $x_1=0, x_2=-0.5, x_3=1$ erreicht. Siehe hierfür \ref{lst:main}
            und \ref{lst:simpAlg}.
        \item
            Durch Einfügen von Slack-Variablen $s_1, s_2$ folgt:
            \begin{eqnarray}
                A &=& \begin{pmatrix}
                    3 & 1 & 1 & 0 \\
                    1 & 2 & 0 & 1
                \end{pmatrix} \\
                x &=& \begin{pmatrix}
                    x_1 \\ x_2 \\ s_1 \\ s_2 
                \end{pmatrix} \\
                b &=& \begin{pmatrix}
                    8 \\ 9
                \end{pmatrix} \\
                c^\text{T} &=& \begin{pmatrix}
                    4 & 5 & 0 & 0
                \end{pmatrix}
            \end{eqnarray}
            
            Das Maximum (24.6) wird für $x_1=0, x_2=1$ erreicht. Siehe hierfür \ref{lst:main}
            und \ref{lst:simpAlg}.
        \item
            \lstinputlisting[basicstyle=\scriptsize,language=Matlab,caption=\texttt{simpAlg.m},label=lst:simpAlg]{simpAlg.m}
            \lstinputlisting[basicstyle=\scriptsize,language=Matlab,caption=\texttt{main.m},label=lst:main]{main.m}
    \end{enumerate}
\end{document}

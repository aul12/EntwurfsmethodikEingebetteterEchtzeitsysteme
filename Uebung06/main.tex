\documentclass[DIN, pagenumber=false, fontsize=11pt, parskip=half]{scrartcl}

\usepackage{amsmath}
\usepackage{amsfonts}
\usepackage{amssymb}
\usepackage{enumitem}
\usepackage[utf8]{inputenc} % this is needed for umlauts
\usepackage[T1]{fontenc} 
\usepackage[ngerman]{babel}
\usepackage{commath}
\usepackage{xcolor}
\usepackage{booktabs}
\usepackage{float}
\usepackage{tikz-timing}
\usepackage{tikz}
\usepackage{multirow}
\usepackage{colortbl}
\usepackage{xstring}
\usepackage{circuitikz}
\usepackage{listings} % needed for the inclusion of source code
\usepackage[final]{pdfpages}
\usepackage{subcaption}
\usepackage{import}
\usepackage{cleveref}
\usepackage{bm}
\usepackage{url}
\usepackage{pgfplots}
\usepackage{mathtools}
\usepackage{pgfgantt}
\usepackage{rotating}

\DeclarePairedDelimiter\ceil{\lceil}{\rceil}
\DeclarePairedDelimiter\floor{\lfloor}{\rfloor}


\usetikzlibrary{calc,shapes.multipart,chains,arrows}

\newcommand{\Prb}[1]{P(\text{#1})}
\newcommand{\CPr}[2]{P(\text{#1}|\text{#2})}
\DeclareMathOperator*{\argmax}{arg\,max}
\DeclareMathOperator*{\argmin}{arg\,min}
\DeclareMathOperator{\rank}{rank}
\newcommand{\R}[0]{\mathbb{R}}

%Inkscape fuckery
\newcommand{\incfig}[2][\columnwidth]{%
    \def\svgwidth{#1}
    \import{./}{#2.eps_tex}
}

\title{Entwurfsmethodik Eingebetteter Echtzeitsysteme}
\author{Paul Nykiel}

\begin{document}
    \maketitle
    \section{Ereignissequenz}
    

    \section{Gantt-Chart}
    \subsection{Rate Monotonic Scheduling}
    Siehe Abbildung \ref{fig:rms}. Hierbei hat $\tau_1$ die kleinste Periode, und damit
    die höchste Priorität, gefolgt von $\tau_2$ und $\tau_3$.

    \begin{sidewaysfigure}
        \centering
        \begin{ganttchart}{0}{39}
            \gantttitlelist{0,1,...,39}{1} \\
            \ganttbar{$\tau_1$}{0}{2}
            \ganttbar{}{10}{12}
            \ganttbar{}{20}{22}
            \ganttbar{}{30}{32}
            \\
            \ganttbar{$\tau_2$}{3}{4} 
            \ganttbar{}{15}{16} 
            \ganttbar{}{33}{34} 
            \\
            \ganttbar{$\tau_3$}{5}{9}
            \ganttbar{}{13}{14}
            \ganttbar{}{17}{17}
            \ganttbar{}{23}{29}
            \ganttbar{}{35}{35}
        \end{ganttchart}
        \caption{}
        \label{fig:rms}
    \end{sidewaysfigure}

    \subsection{Deadline Monotonic Scheduling}
    Siehe Abbildung \ref{fig:rms}. Hierbei hat $\tau_1$ die kürzeste Deadline, und damit
    die höchste Priorität, gefolgt von $\tau_2$ und $\tau_3$. Folglich gilt das in diesem
    Fall RMS und DMS das selbe Ergebnis liefern.

    \subsection{Earliest Deadline First}
    Siehe Abbildung \ref{fig:edf}.

    \begin{sidewaysfigure}
        \centering
        \begin{ganttchart}{0}{39}
            \gantttitlelist{0,1,...,39}{1} \\
            \ganttbar{$\tau_1$}{0}{2}
            \ganttbar{}{10}{12}
            \ganttbar{}{20}{22}
            \ganttbar{}{30}{32}
            \\
            \ganttbar{$\tau_2$}{3}{4} 
            \ganttbar{}{16}{17}
            \ganttbar{}{34}{35}
            \\
            \ganttbar{$\tau_3$}{5}{9}
            \ganttbar{}{13}{15}
            \ganttbar{}{23}{29}
            \ganttbar{}{33}{33}
        \end{ganttchart}
        \caption{}
        \label{fig:edf}
    \end{sidewaysfigure}
\end{document}

\documentclass[DIN, pagenumber=false, fontsize=11pt, parskip=half]{scrartcl}

\usepackage{amsmath}
\usepackage{amsfonts}
\usepackage{amssymb}
\usepackage{enumitem}
\usepackage[utf8]{inputenc} % this is needed for umlauts
\usepackage[T1]{fontenc} 
\usepackage[ngerman]{babel}
\usepackage{commath}
\usepackage{xcolor}
\usepackage{booktabs}
\usepackage{float}
\usepackage{tikz-timing}
\usepackage{tikz}
\usepackage{multirow}
\usepackage{colortbl}
\usepackage{xstring}
\usepackage{circuitikz}
\usepackage{listings} % needed for the inclusion of source code
\usepackage[final]{pdfpages}
\usepackage{subcaption}
\usepackage{import}
\usepackage{cleveref}
\usepackage{bm}
\usepackage{url}
\usepackage{pgfplots}
\usepackage{mathtools}
\usepackage{pgfgantt}
\usepackage{rotating}

\DeclarePairedDelimiter\ceil{\lceil}{\rceil}
\DeclarePairedDelimiter\floor{\lfloor}{\rfloor}


\usetikzlibrary{calc,shapes.multipart,chains,arrows}

\newcommand{\Prb}[1]{P(\text{#1})}
\newcommand{\CPr}[2]{P(\text{#1}|\text{#2})}
\DeclareMathOperator*{\argmax}{arg\,max}
\DeclareMathOperator*{\argmin}{arg\,min}
\DeclareMathOperator{\rank}{rank}
\newcommand{\R}[0]{\mathbb{R}}

%Inkscape fuckery
\newcommand{\incfig}[2][\columnwidth]{%
    \def\svgwidth{#1}
    \import{./}{#2.eps_tex}
}

\title{Entwurfsmethodik Eingebetteter Echtzeitsysteme}
\author{Paul Nykiel}

\begin{document}
    \maketitle
    \section{Antwortzeit}
    Für diese Aufgabe wurde der im Skript in Satz 20 vorgestellte Algorithmus mithilfe von Python implementiert,
    siehe hierfür auch Abschnitt \ref{sec:app:response}.

    \subsection{$\Gamma_2$}
    Antwortzeiten für das erste Taskset:
    \begin{table}[H]
        \centering
        \begin{tabular}{cc}
            \toprule
                Task & $r_i$ \\ 
            \midrule
                $\tau_1$ & 1 \\
                $\tau_2$ & 3 \\
                $\tau_3$ & 7 \\
                $\tau_4$ & 18 \\
            \bottomrule
        \end{tabular}
    \end{table}

    \subsection{$\Gamma_3$}
    Antwortzeiten für das zweite Taskset:
    \begin{table}[H]
        \centering
        \begin{tabular}{cc}
            \toprule
                Task & $r_i$ \\ 
            \midrule
                $\tau_1$ & 1 \\
                $\tau_2$ & 3 \\
                $\tau_3$ & 10 \\
            \bottomrule
        \end{tabular}
    \end{table}


    \section{Busy Period}
    Für die Busy Period kann ein Taskset bis zur Hyperperiode berechnet werden und dann
    die längste Zeit ohne Task bestimmt werden.

    Hierfür wurde ebenfalls ein Python Skript entwickelt, siehe Abschnitt \ref{sec:app:busy}.

    Die maximale Busy Period für $\Gamma_4$ beträgt 23, für $\Gamma_5$ 17.

    \appendix
    \section{Antwortzeitanalyse} \label{sec:app:response}
    \lstinputlisting[language=python, caption=\texttt{response.py}, label=lst:response]{response.py}

    \section{Busy Period} \label{sec:app:busy}
    \lstinputlisting[language=python, caption=\texttt{busy.py}]{busy.py}
\end{document}

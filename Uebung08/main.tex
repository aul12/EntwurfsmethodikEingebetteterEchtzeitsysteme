\documentclass[DIN, pagenumber=false, fontsize=11pt, parskip=half]{scrartcl}

\usepackage{amsmath}
\usepackage{amsfonts}
\usepackage{amssymb}
\usepackage{enumitem}
\usepackage[utf8]{inputenc} % this is needed for umlauts
\usepackage[T1]{fontenc} 
\usepackage[ngerman]{babel}
\usepackage{commath}
\usepackage{xcolor}
\usepackage{booktabs}
\usepackage{float}
\usepackage{tikz-timing}
\usepackage{tikz}
\usepackage{multirow}
\usepackage{colortbl}
\usepackage{xstring}
\usepackage{circuitikz}
\usepackage{listings} % needed for the inclusion of source code
\usepackage[final]{pdfpages}
\usepackage{subcaption}
\usepackage{import}
\usepackage{cleveref}
\usepackage{bm}
\usepackage{url}
\usepackage{pgfplots}
\usepackage{mathtools}
\usepackage{pgfgantt}
\usepackage{rotating}

\DeclarePairedDelimiter\ceil{\lceil}{\rceil}
\DeclarePairedDelimiter\floor{\lfloor}{\rfloor}


\usetikzlibrary{calc,shapes.multipart,chains,arrows}

\newcommand{\Prb}[1]{P(\text{#1})}
\newcommand{\CPr}[2]{P(\text{#1}|\text{#2})}
\DeclareMathOperator*{\argmax}{arg\,max}
\DeclareMathOperator*{\argmin}{arg\,min}
\DeclareMathOperator{\rank}{rank}
\newcommand{\R}[0]{\mathbb{R}}

%Inkscape fuckery
\newcommand{\incfig}[2][\columnwidth]{%
    \def\svgwidth{#1}
    \import{./}{#2.eps_tex}
}

\title{Entwurfsmethodik Eingebetteter Echtzeitsysteme}
\author{Paul Nykiel}

\begin{document}
    \maketitle
    \section{Antwortzeit}
    \begin{enumerate}[label=\alph*)]
        \item Für die allgemeine Antwortzeitanalyse nach Theorem 8 müssen alle Tasks höherer Priorität betrachtet
            werden. Da die Prioritäten auf Basis von DMS festgelegt werden müssen also alle Tasks mit kürzerer Deadline
            betrachtet werden. Das heißt auf CPU 1 ist $\tau_3$, auf CPU 2 $\tau_5$ jeweils die Task mit größter 
            Priorität. Für diese muss die Antwortzeit nicht mithilfe des Busy Windows Verfahrens bestimmt werden.

            Bei $\tau_2$ ist die Deadline gleich der maximalen Ausführungszeit
            daher muss dort auch nicht das Busy Windows Verfahren angewandt werden.
        \item 
            Für die Aufgabe wurde ein Python Skript implementiert, siehe Anhang \ref{app:sec:response}.

            Die maximalen Antwortzeiten Betragen:
            \begin{eqnarray}
                r^+_{\tau_1} &=& 11 \\
                r^+_{\tau_2} &=& 3 \\
                r^+_{\tau_3} &=& 1 \\
                r^+_{\tau_4} &=& 6 \\
                r^+_{\tau_5} &=& 2
            \end{eqnarray}
    \end{enumerate}

    \section{Antwortzeit}
    Für die Aufgabe wurde ein Python Skript implementiert, siehe Anhang \ref{sec:app:jitter}.
    \begin{enumerate}[label=\alph*)]
        \item Der Jitter beträgt:
            \begin{eqnarray}
                J_4 &=& 27 \\
                J_5 &=& 1
            \end{eqnarray}
        \item Die Minimale Antwortzeit beträgt:
            \begin{equation}
                r^-_{\tau_4} = 320
            \end{equation} 
    \end{enumerate}

    \appendix
    \section{Aufgabe 1} \label{app:sec:response}
    \lstinputlisting[language=Python, caption=\texttt{response.py}]{response.py}

    \section{Aufgabe 2} \label{app:sec:jitter}
    \lstinputlisting[breaklines=True, language=Python, caption=\texttt{jitter.py}]{jitter.py}
    
\end{document}

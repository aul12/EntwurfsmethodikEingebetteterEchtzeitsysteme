\documentclass[DIN, pagenumber=false, fontsize=11pt, parskip=half]{scrartcl}

\usepackage{amsmath}
\usepackage{amsfonts}
\usepackage{amssymb}
\usepackage{enumitem}
\usepackage[utf8]{inputenc} % this is needed for umlauts
\usepackage[T1]{fontenc} 
\usepackage[ngerman]{babel}
\usepackage{commath}
\usepackage{xcolor}
\usepackage{booktabs}
\usepackage{float}
\usepackage{tikz-timing}
\usepackage{tikz}
\usepackage{multirow}
\usepackage{colortbl}
\usepackage{xstring}
\usepackage{circuitikz}
\usepackage{listings} % needed for the inclusion of source code
\usepackage[final]{pdfpages}
\usepackage{subcaption}
\usepackage{import}
\usepackage{cleveref}
\usepackage{bm}
\usepackage{url}
\usepackage{pgfplots}
\usepackage{mathtools}
\usepackage{pgfgantt}
\usepackage{rotating}

\DeclarePairedDelimiter\ceil{\lceil}{\rceil}
\DeclarePairedDelimiter\floor{\lfloor}{\rfloor}


\usetikzlibrary{calc,shapes.multipart,chains,arrows}

\newcommand{\Prb}[1]{P(\text{#1})}
\newcommand{\CPr}[2]{P(\text{#1}|\text{#2})}
\DeclareMathOperator*{\argmax}{arg\,max}
\DeclareMathOperator*{\argmin}{arg\,min}
\DeclareMathOperator{\rank}{rank}
\newcommand{\R}[0]{\mathbb{R}}

%Inkscape fuckery
\newcommand{\incfig}[2][\columnwidth]{%
    \def\svgwidth{#1}
    \import{./}{#2.eps_tex}
}

\title{Entwurfsmethodik Eingebetteter Echtzeitsysteme}
\author{Paul Nykiel}

\begin{document}
    \maketitle
    \section{Auslastungsanalyse und Testgrenzen}
    \begin{enumerate}[label=\alph*)]
        \item
            Es gilt:
            \begin{equation}
                U = \sum_{\tau \in \Gamma} \frac{c^+_\tau}{p_\tau}
                    = \frac{5}{10} + \frac{10}{55} + \frac{15}{100} + \frac{20}{200}
                    = 0.93
            \end{equation}
            da $U = 0.93 \leq 1$ kann die Taskmenge nach Fristen geplanten werden.
        \item
            Die Hyperperiode beträgt:
            \begin{equation}
                H = \text{lcm}(10, 55, 100, 200) = 2200
            \end{equation}
            
            Testgrenze nach Baruah:
            \begin{equation}
                h'_\tau = \frac{U}{1-U} \max_{\tau \in \Gamma} \left(p_\tau - d_\tau\right)
                        = \frac{0.93}{1-0.93} (200-150) < 684
            \end{equation} 

            Testgrenze nach Ripoll:
            \begin{equation}
                h''_\tau = \frac{\sum_{\tau \in \Gamma} (p_\tau - d_\tau) \frac{c_\tau}{p_\tau}}{1 - U} = \frac{13.98}{1-0.93} < 206
            \end{equation}

            Testpunkte bis zur Hyperperiode:
            \begin{equation}
                n_\text{Hyper} = \frac{2200}{5} = 440
            \end{equation}

            Testpunkte bis zur Testgrenze nach Baruah:
            \begin{equation}
                n_\text{Baruah} = \frac{684}{5} < 137
            \end{equation}

            Testpunkte bis zur Testgrenze nach Ripoll:
            \begin{equation}
                n_\text{Ripoll} = \frac{206}{5} < 42
            \end{equation}

            Es wird die Testgrenze nach Ripoll gewählt. Für die Überprüfung der Rechnenzeit
            wurde ein Python Skript implementiert, siehe hierfür Anhang 
            \ref{sec:app:rechenzeit}. Der Test ergibt, dass die Taskmenge geplannt werden
            kann.
    \end{enumerate}

    \section{Approximation durch straffen der Periode}
    \begin{enumerate}[label=\alph*)]
        \item
            Die Taskperioden sind teilerfremd, daher ist die Hyperperiode das Produkt
            der einzelnen Perioden
            \begin{equation}
                H = \prod_{\tau \in \Gamma} p_\tau = 43890
            \end{equation}
            da die Perioden Teilerfremd sind werden genauso viele Testpunkte benötigt.
        \item
            Wähle $p_3$ zu $55$, dann gilt:
            \begin{equation}
                \tilde{H} = 770
            \end{equation}
            Da $22$ und $35$ weiterhin teilerfremd sind werden auch $720$ Testpunkte
            benötigt.

            Für die Analyse wurde das Python Skript aus der vorherigen Aufgabe modizifiert,
            siehe Anhang \ref{sec:app:rechenzeit2}. Der Test ergibt, dass die Taskmenge
            geplannt werden kann.
        \item
            Es gilt $d \leq p$. Mit Satz 12 gilt dann:
            \begin{eqnarray}
                && \sum_{\tau \in \Gamma} \floor*{\frac{\Delta t - d_\tau}{p_\tau} + 1} c^+_\tau\\
                &\leq& \sum_{\tau \in \Gamma} \floor*{\frac{\Delta t - d_\tau}{d_\tau} + 1} c^+_\tau\\
                &=& \sum_{\tau \in \Gamma} \floor*{\frac{\Delta t}{d_\tau} - 1 + 1} c^+_\tau\\
                &=& \sum_{\tau \in \Gamma} \floor*{\frac{\Delta t}{d_\tau}} c^+_\tau\\
                &\leq& \sum_{\tau \in \Gamma} \frac{\Delta t}{d_\tau} c^+_\tau\\
                &=& \Delta t \sum_{\tau \in \Gamma} \frac{c^+_\tau}{d_\tau}\\
            \end{eqnarray}

            Zudem gilt:
            \begin{equation}
                \sum_{\tau \in \Gamma} \frac{c^+_\tau}{d_\tau} \leq 1 \Leftrightarrow
                \Delta t \sum_{\tau \in \Gamma} \frac{c^+_\tau}{d_\tau} \leq \Delta t
            \end{equation}
            D.h.
            \begin{equation}
                \sum_{\tau \in \Gamma} \frac{c^+_\tau}{d_\tau} \leq 1 \Rightarrow
                \sum_{\tau \in \Gamma} \floor*{\frac{\Delta t - d_\tau}{p_\tau} + 1} c^+_\tau < 
                \Delta t
            \end{equation}
            Damit ist ein notwendiger Test, der in konstanter Zeit berechnet werden kann gefunden.

            Für das gegebene Taskset gilt:
            \begin{equation}
                \sum_{\tau \in \Gamma} \frac{c^+_\tau}{d_\tau} = 1 \leq 1
            \end{equation}
            folglich ergibt der Test, dass das Taskset planbar ist.
    \end{enumerate} 


    \appendix
    \section{Rechenzeit} \label{sec:app:rechenzeit}
    \lstinputlisting[language=Python,caption=\texttt{rechenzeit.py}]{rechenzeit.py}

    \section{Rechenzeit 2} \label{sec:app:rechenzeit2}
    \lstinputlisting[language=Python,caption=\texttt{rechenzeit2.py}]{rechenzeit2.py}
\end{document}
